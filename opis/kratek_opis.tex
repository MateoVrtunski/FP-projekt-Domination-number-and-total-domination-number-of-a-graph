\documentclass[a4paper,12pt]{article}
\usepackage[slovene]{babel}
\usepackage[utf8]{inputenc}
\usepackage[T1]{fontenc}
\usepackage{lmodern}
\usepackage{amsmath}
\usepackage{amssymb}
\usepackage[shortlabels]{enumitem}
\usepackage{graphicx}

\newtheorem{definition}{Definicija}

\pagestyle{plain}

\begin{document}
\author{Mateo Vrtunski in Aljaž Kump}
\date{December 2023}
\title{Dominacijsko in popolno dominacijsko število grafa - kratek opis}
\maketitle

\section{Opis}

State the problems of determining the domination number $\gamma (G)$ and the total domination number $\gamma_t (G)$ of a graph as ILPs. Also consider their LP relaxations.
Experimentally study the differences between these four numbers for various classes of graphs.
Try to determine some bounds on these numbers for trees, hypercubes, Kneser graphs,\ldots  with
a given number of vertices n. If your programming environment already has implementations of
these two invariants, then compare the speed of your program with the ones provided. Report your results.

\section{Definicije}

    \begin{definition}
       Množica $S\subseteq V$ vozlišč grafa $G = (V,E)$ je \textbf{dominacijska množica}, če je vsako vozlišče $v \in  V$ element množice $S$ ali pa je sosednje kakemu elementu množice $S$.
    \end{definition}

    \begin{definition}
        \textbf{Dominacijsko število grafa} $\gamma(G)$ je enako velikosti minimalne dominacijske množice grafa $G$, katere je moč najmanjša. 
    \end{definition}

    \begin{definition}
        Množica $S$ vozlišč grafa $G$ je \textbf{popolna dominacijska množica}, če je vsako vozlišče $v \in  S$ izolirano od $S$, vsako vozlišče $v \in V/S$ pa je sosed natanko enemu vozlišču iz $S$.
    \end{definition}
\pagebreak

\section{Problem} 
    Za različne grafe morava ugotoviti kako izračunati njihovo dominacijsko število $\gamma(G)$ in popolno dominacijsko število  $\gamma_t (G)$ ter odgovarjala na vprašanja kot so:  "Ali je $\gamma(G)$ enolično določen za vsak graf?"
    
\section{Načrt dela}
    Napisala bova linearni program, ki nam bo poiskal $\gamma(G)$ in $\gamma_t (G)$. Uporabila bova tudi relaksacijo ter primerjala dobljene rezultate med sabo. 
    Primerjala bova tudi najine rezultate z implementirano funkcijo v Sage-u.

\end{document}